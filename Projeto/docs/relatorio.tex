% !TeX spellcheck = pt_PT
\documentclass[a4paper]{report}
\usepackage[portuguese]{babel}
\usepackage{a4wide}

\usepackage{graphicx}
\usepackage{hyperref}
\usepackage{listings}
\usepackage{indentfirst}
\usepackage{float}

\setlength{\parskip}{1em}

\title{POO - Trabalho Prático\\
	\large Grupo 05}
\author{João Barbosa (a93270)
	\and Simão Cunha (a93262)
	\and Tiago Silva (a93277)}
\date{Ano Letivo 2020/2021}

\begin{document}
	\begin{minipage}{0.9\linewidth}
        \centering
		\includegraphics[width=0.4\textwidth]{eng.jpeg}\par\vspace{1cm}
		\href{https://www.uminho.pt/PT}
		{\scshape\LARGE Universidade do Minho} \par
		\vspace{0.6cm}
		\href{https://miei.di.uminho.pt/}
		{\scshape\Large Mestrado Integrado em Engenharia Informática} \par
		\maketitle
		\begin{figure}[H]
			\includegraphics[width=0.32\linewidth]{joao.png}
			\includegraphics[width=0.32\linewidth]{simao.png}
			\includegraphics[width=0.32\linewidth]{tiago.png}
		\end{figure}
	\end{minipage}
	
	\tableofcontents
	
	\pagebreak
	
	\chapter{Introdução e principais desafios}
%
	Este projeto consistiu em criar um sistema de gestão e simulação de equipas de um determinado
	desporto (neste caso futebol) muito semelhante ao jogo \textit{Football Manager}, de forma a
	aplicarmos os vários conhecimentos adquiridos nas aulas ao longo do semestre.
	
	Tivemos alguns obstáculos para a concessão deste projeto, como a falta de tempo para conseguir uma aplicação que respeitasse melhor os princípios de programação orientada a objetos.
	
	\chapter{Classes}
	\section{Main}
	Classe responsável pelo arranque do programa. Para isto, o método \texttt{main} invoca o método \texttt{run} da classe \texttt{Controlo}.
	
	\section{Controlo}
	\begin{lstlisting}[language=Java]
    private ControloDados cd;
    private Scanner scan;
    private final String[] menuPrincipal;
    private final String[] menuCarregarDados;
    private final String[] menuSimulacao;
    private final String[] menuEditarEquipa;
    private final String[] menuEditarJogador;
    private final String[] menuCriarDados;
    private final String[] menuCriarJogador;
    private final String[] menuPosicao;
    private final String[] menuEscolheTatica;
    
	\end{lstlisting}
	\texttt{Controlo} é a classe responsável pelo controlo da aplicação. Associa um menu a uma funcionalidade do programa. Por exemplo, mostra no ecrã a simulação de um jogo de futebol, recorrendo a métodos da classe \texttt{ExecutaPartida} e com mensagens pré-concebidas para vários momentos do jogo. 
	
	 Neste menu temos as seguintes opções: \par
	 \vspace{0.05cm}
1. \textit{Simular uma partida de futebol}: é possível escolher equipas, definir o seu onze inicial, esquema tático, efetuar um jogo, etc \par
2. \textit{Manipular dados}: é possível carregar, criar, ver/editar, gravar e fazer reset a dados

      
	
	\section{ControloDados}
	\begin{lstlisting}[language=Java]
private Map<String,EquipaFutebol>equipas;//nome + equipa
private Map<String,List<PartidaFutebol>>partidas;//Nome + Lista dos jogos 
	\end{lstlisting}
	
\texttt{Controlo de dados} é a classe de controlo para a execução de partidas de futebol, que inclui uma \texttt{HashMap} de equipas de futebol que são identificadas pelo seu nome e um \texttt{HashMap} de partidas de futebol onde é possivel associar o nome de uma equipa à lista de todos os seus jogos efetuados.
 
	
	\section{ViewJogo}
	\begin{lstlisting}[language=Java]
    private static Scanner is;
    private boolean continuacao;
    private List<String> opcoes;        
    private List<PreCondition> disponivel; 
    private List<Handler> handlers;     
	\end{lstlisting}
	A classe \texttt{ViewJogo} gere o fluxo do programa através da gestão de menus. Esta é a classe responsável por gerir o menu inicial, disponível para todas as entidades como \textit{JogoFutebol} ou \textit{EquipaFutebol}, que interage com outras classes controladoras que gerem os menus de cada entidade em particular.

    \section{Jogador}
    \begin{lstlisting}[language=Java]
    private int numero; 
    private String nome; 
    private String posicao; 
    private List <String> historial; 
    private Atributos atributos; 
    \end{lstlisting}
    
      Esta é uma das classes essenciais da aplicação. A classe \texttt{jogador} representa um jogador de futebol, onde são armazenados o número da camisola, o seu nome, a sua posição em campo, o historial dos clubes por onde passou e os seus atributos (como a capacidade de remate, velocidade, passe,...). Existem \textbf{5} tipos de jogador: guarda-redes, defesa, lateral, médio e avançado.
      
      
    \section{CompPosicao}
    
    A classe \texttt{CompPosicao} implementa um \textit{Comparator} para a classe \texttt{Jogador}.
    Este \texttt{Comparator} compara dois jogadores através dos números da camisola. 
    
	\section{Atributos}
	\begin{lstlisting}[language=Java]
    private int velocidade;
    private int resistencia;
    private int destreza;
    private int impulsao;
    private int jogoDeCabeca;
    private int remate;
    private int controloDePasse;
    \end{lstlisting}
    
    A classe \texttt{Atributos} é abstrata, pois os valores atribuídos para cada parâmetro variam conforme a posição do jogador. Como será visível mais à frente, decidimos adicionar mais atributos 
    a cada tipo de jogador, mais própriamente em classe hereditárias. 
	
	\subsection{AtributosGR}
	\begin{lstlisting}[language=Java]
    private int elasticidade;
    private int reflexos;
    \end{lstlisting}
    
    Esta classe herda atributos da classe \texttt{Atributos} e decidimos acrescentar mais 2 atributos: a elasticidade e os reflexos.
    
    \subsection{AtributosLateral}
    \begin{lstlisting}[language=Java]
    private int precisaoCruzamentos;
    private int drible;
    \end{lstlisting}

    Esta classe herda atributos da classe \texttt{Atributos} e decidimos acrescentar mais 2 atributos: a precisão em efetuar cruzamentos e a capacidade de driblar um adversário.
	
	\subsection{AtributosDefesa}
    \begin{lstlisting}[language=Java]
    private int posicionamentoDefensivo;
    private int cortes;
    \end{lstlisting}
    
    Esta classe herda atributos da classe \texttt{Atributos} e decidimos acrescentar mais 2 atributos: o posicionamento defensivo e a capacidade de cortar a bola.
        
        
    \subsection{AtributosMedio}
    \begin{lstlisting}[language=Java]
    private int recuperacaoDeBolas;
    private int visaoDeJogo;
    \end{lstlisting}
    
    Esta classe herda atributos da classe \texttt{Atributos} e decidimos acrescentar mais 2 atributos: a capacidade de recuperação da bola e a visão de jogo.
    
    \subsection{AtributosAvancado}
    \begin{lstlisting}[language=Java]
    private int penaltis;
    private int desmarcacao;
    \end{lstlisting}
    
    Esta classe herda atributos da classe \texttt{Atributos} e decidimos acrescentar mais 2 atributos:
    a capacidade de marcação da marca dos onze metros e a desmarcação para se isolar dos adversários.    
	\section{Plantel}
	
	\begin{lstlisting}[language=Java]
    private Map<Integer,Jogador> titulares; 
    private Map<Integer,Jogador> suplentes; 
    private int nJogadoresNoPlantel; 
    private Tatica tatica; 
	\end{lstlisting}
    
    A classe \texttt{Plantel} representa um plantel de futebol, com um \textit{map} de jogadores no onze inicial,
    um \textit{map} de jogadores no banco, o número total de jogadores num plantel e o esquema tático adotado pela equipa.
	 
	
	\section{Tatica}
	\begin{lstlisting}[language=Java]
    private int nGR;
    private int nDF;
    private int nLT;
    private int nMD;
    private int nPL;
	\end{lstlisting}
    A classe \texttt{Tatica} define o esquema tático de uma equipa. Para tal, será necessário saber o número de guarda-redes (que será sempre 1), o número de defesas, o número de laterais, o número de médios e o número de avançados
	
	\section{EquipaFutebol}
	\begin{lstlisting}[language=Java]
    private String nome;
    private Plantel plantel;
	\end{lstlisting}
    O conceito da classe \texttt{EquipaFutebol} pode ser confundido com o conceito da classe
    \texttt{Plantel}. 
    Uma \texttt{Plantel} é dividido em vários parâmetros para serem mais facilmente acedidos pelo
    utilizado. 
    Já uma \texttt{EquipaFutebol} engloba todos estes parâmetros num objeto \texttt{Plantel}, incluindo o nome da equipa.
	
	
	
	\section{PartidaFutebol}
	\begin{lstlisting}[language=Java]
    private double tempo;
    private int golosVisitante;
    private int golosVisitado;
    private int substituicoesVisitante, substituicoesVisitados;
    private int[][] subsVisitante, subsVisitada;
    private EquipaFutebol equipaVisitante, equipaVisitada;
    private LocalDate data;
};
   
	\end{lstlisting}
    A classe \texttt{PartidaFutebol} representa uma partida de futebol. São incluídos o tempo de jogo, o número de golos da equipa visitante, o número de golos da equipa da casa, o número de substituições que a equipa visitante pode efetuar, o número de substituições da equipa da casa que ainda pode fazer, um \textit{array} de substituições de uma equipa (visitante ou visitada) onde cada jogador é representado
    pelo seu número de camisola, os equipas que vão jogar e a data da realização do jogo.
	  
		\section{ExecutaPartida}
	\begin{lstlisting}[language=Java]
    private PartidaFutebol partida;
    private Jogador jogadorAtual;
    private Boolean casa; 
    private Boolean comecou;
    private Random random;
    private static final double acaoRapida = 0.5;
    private static final double acaoMedia = 1.0;
	
	\end{lstlisting}
		A classe \texttt{ExecutaPartida} é responsável por, tal como diz o nome, executar uma partida de futebol.
		É constituído por uma partida de futebol, pelo jogador que possui atualmente a bola, um booleano que verifica se o jogador é da equipa adversária, um booleano que verifica se um jogo já começou, um objeto \texttt{Random} para gerar números aleatórios (por exemplo) e duas constantes que simbolizam uma \textit{ação rápida} (por exemplo, tentar rematar a bola) e uma \textit{ação média} (por exemplo, marcar golo).
		

	\chapter{Estrutura do projeto}
	
	O nosso projeto segue a estrutura \textit{Model View Controller} (MVC), estando por isso organizado em três camadas:
	\begin{itemize}
		\item A camada de dados (o \textbf{"Model"}) é composta pelas classes Jogador, Atributos, AtributosGR, AtributosDefesa, AributosLateral, AtributosMedio, AtributosAvancado, EquipaFutebol, Plantel, Tatica e PartidaFutebol e pela interface CompPosicao.
		\item A camada de interação com o utilizador (a \textbf{"View"}, ou apresentação) é composta unicamente pela classe ViewJogo, embora existam algumas mensagens impressas para o \textit{stdout} noutras classes de outras camadas.
		\item A camada de controlo do fluxo do programa (o \textbf{"Controller}) é composta pela classe ExecutaPartida, ControloDados. Controlo e pelo Main.
	\end{itemize}
	
	Todo este projeto cumpriu um princípio principal da programação orientada aos objetos: o \textit{encapsulamento}. São exemplos notórios deste cumprimento os \textit{getters} e \textit{setters} de todas as classes tanto para objetos imutáveis (por exemplo, \textit{string} ou \textit{integer}) como para objetos mutáveis (por exemplo \textit{Jogador} ou \textit{Equipa}).
    
      

	
	

	\chapter{Conclusão}

	A nível geral, e tendo em conta o que foi explicado nos capítulos anteriores, podemos afirmar que temos um projeto bem conseguido, apesar de não estar exatamente igual ao que idealizávamos. Acreditamos que respondemos de forma correta à simulação de uma partida de futebol, que foi um dos pontos mais trabalhosos do nosso projeto. Além disso, é possível um crescimento controlado da nossa aplicação, uma vez que é possível adicionar mais desportos no diretório \textit{Desporto} do nosso projeto.
	
	\appendix
	
	\chapter{Diagrama de Classes}
	\begin{figure}[H]
		\begin{center}
			\includegraphics[height=0.6\textheight]{diagrama.png}
			\caption{Diagrama de classes do programa, gerado pelo \emph{IntelliJ}}
		\end{center}
	\end{figure}
\end{document}